\section{Discussion}

\paragraph{What is established by completed evidence.}
The project currently supports: (i) hidden-state safety probing is feasible, (ii) stop-only safety can underperform continuous correction in strict zero-shot OOD settings, and (iii) steering has favorable controller-step latency versus planning baselines.

\paragraph{What remains unresolved.}
Whether steering consistently exceeds MPPI across all seeds/splits/models remains an empirical question pending completion of queued OpenVLA OOD runs.

\paragraph{Embodied-track status.}
Yahboom-arm integration is in-scope and instrumented at the manuscript level, but final embodied claims are pending completion of physical trials with standardized safety and trial-accounting protocols.

\paragraph{Failure modes observed.}
Over-intervention and threshold mismatch can degrade outcomes. Results are sensitive to gating and clamp parameters, motivating conservative shared profiles and multi-seed reporting.

\paragraph{Claim boundary for camera-ready.}
This work should claim transferable latent safety structure with current evidence; a stronger universal-manifold claim should be conditioned on final OOD seed-complete statistics.

\paragraph{Placeholders for final discussion.}
TODO-D1: Robustness-to-seed section. \\
TODO-D2: Task-family sensitivity section. \\
TODO-D3: Latent dimensionality and recoverability section.
